%
% Documento: Introdução
%

\chapter{Introdução}\label{chap:introducao}

Desde que a \textit{World Wide Web} foi proposta pelo cientista e pesquisador britânico Tim Bernes-Lee em 1989 \cite{WebHistory}, as páginas \textit{web} vêm mudando de maneira acelerada como pode ser percebido observando os gráficos nas \autoref{fig:httpcontenttype2011} e \autoref{fig:httpcontenttype2015}, gerados com a ajuda do \textit{website} HTTP Archive \footnote{http://httparchive.org/}. O primeiro foi gerado com dados de 15 de Abril de 2011 e o segundo com dados de 15 de Abril de 2015 e os dois mostram a média de \textit{bytes} por página por tipo de conteúdo nas páginas \textit{web}. Mas a informação mais relevante está localizado na parte debaixo do gráfico e mostra o tamanho médio de uma página \textit{web} nas respectivas dadas.

\begin{figure}[!htb]
    \centering
    \caption{Média de Bytes por Página por Tipo de Conteúdo em 2011}
    \includegraphics[width=1.0\textwidth]{./04-figuras/introducao/bytes_content_type_april_2011}
    \fonte{Adaptado de \citeonline{HttpArchiveContentType2011}}
    \label{fig:httpcontenttype2011}
\end{figure}

\begin{figure}[!htb]
    \centering
    \caption{Média de Bytes por Página por Tipo de Conteúdo em 2015}
    \includegraphics[width=1.0\textwidth]{./04-figuras/introducao/bytes_content_type_april_2015}
    \fonte{Adaptado de \citeonline{HttpArchiveContentType2015}}
    \label{fig:httpcontenttype2015}
\end{figure}

Nos últimos 4 anos, o tamanho médio de uma página \textit{web} passou de 769kB para 2061kB, um expressivo aumento de 168\%. Apesar dessa grande mudança no tamanho das páginas (e consequentemente dos \textit{websites}), a maneira como \textit{websites} são entregues dos servidores para os clientes não sofreu nenhuma alteração desde 1999, ano de lançamento da RFC 2616 que especificou o HTTP/1.1 \cite{RFC2616}. Como explicado por \cite{Tanenbaum}, o HTTP é um protocolo simples na camada de aplicação de requisições e respostas que é executado em cima da camada de transporte do protocolo TCP. O HTTP ficou famoso por ser fácil de entender e implementar e ao mesmo tempo cumprir sua função de transferência de recursos em rede com um bom desempenho. Contudo, o aumento no tamanho dos \textit{websites} começou a fazer com que o tempo de resposta das páginas \textit{web} ficasse muito grande e, como mudanças no HTTP seriam muito difíceis, pois teriam de envolver esforços de muitas partes interessadas na \textit{World Wide Web} (como fabricantes de navegadores e mantenedores de servidores), os desenvolvedores passaram a ter de criar outras formas de resolver esse problema.

Técnicas de otimização de desempenho passaram a ser estudadas e implementadas por muitas empresas que queriam ter seus \textit{websites} entregues mais rapidamente a seus clientes. Por muitos anos, a grande maioria dessas empresas focou seus esforços em otimizações para o \textit{back-end}, principalmente para os seus servidores, o que hoje em dia pode ser considerado um erro. Como explicado por \cite[p.~35]{HighPerformance} no que ele chamou de "Regra de Ouro do Desempenho": "Apenas 10-20\% do tempo de resposta do usuário final são gastos baixando o documento HTML. Os outros 80-90\% são gastos baixando todos os componentes da página.".

Dessa forma ficou claro que técnicas de otimização de desempenho para o \textit{front-end} dos \textit{websites} deveriam se tornar prioridade quando procura-se melhorar o tempo de resposta para o usuário final. Para entender o quão importante esse tempo de resposta se tornou para empresas que baseiam seus negócios em vendas de serviços ou produtos na Internet, basta observar os dados da tabela \autoref{tab:impactodesempenho} expostos por Steve Souders na conferência Google I/O de 2009:

\begin{table}[h]
	\centering
	\caption{Impacto do desempenho de \textit{website} na receita.\label{tab:impactodesempenho}}
	\begin{tabular}{lcl}
		\hline
			\textbf{Empresa} & \multicolumn{1}{l}{\textbf{Piora no tempo de resposta}} & \textbf{Consequência}  \\
		\hline
			Google Inc.      & +500ms                                                  & -20\% de tráfego       \\
			Yahoo Inc.       & +400ms                                                  & -5\% a -9\% de tráfego \\
			Amazon.com Inc.  & a cada +100ms                                           & -1\% de vendas         \\           
		\hline
	\end{tabular}
	\fonte{\cite{GoogleIO2009}}
\end{table}

Steve Souders tornou-se um grande evangelizador da área de otimização de desempenho de \textit{front-end} de \textit{websites}. Em seus livros, \textit{High Performance Websites} \cite{HighPerformance} e \textit{Even Faster Websites} \cite{EvenFaster} ele ensina técnicas de como tornar \textit{websites} mais rápidos, focando nos componentes das páginas. E em 2012, ele lançou seu terceiro livro, \textit{Web Performance Daybook} \cite{WebPerformance}, como um guia para desenvolvedores que trabalham com otimização de desempenho de \textit{websites}.

\section{Motivação}
\label{sec:motivacao}

Após mais de 15 anos sem mudanças, o protocolo HTTP (finalmente) receberá uma atualização. A nova versão do protocolo, chamada de HTTP2, teve sua especificação aprovada no dia 11 de Fevereiro de 2015, \cite{HTTP2Spec}, e deverá começar a ser implantada, a partir de 2016. Muitas mudanças foram feitas com o objetivo de melhorar o desempenho e a segurança da \textit{web}. Além disso, o HTTP2 foi desenvolvido para ser compatível com suas versões anteriores, não sendo necessárias mudanças em servidores e aplicações antigos para funcionar baseados no novo protocolo.

Com as novas funcionalidades do HTTP2 a caminho algumas coisas devem mudar na área de otimização de desempenho de \textit{websites}. Como pode ser percebido em \cite{HTTP2Explained}, o HTTP2 foi desenvolvido para melhorar o desempenho de todos os \textit{websites} e aplicações \textit{web}, e não apenas dos poucos que podem aplicar técnicas de otimização. Isso torna difícil uma previsão o resultado da aplicação de técnicas desenvolvidas para os protocolos HTTP/1.0 e HTTP/1.1. Acredita-se que algumas das técnicas antigas podem não apenas não melhorar o desempenho dos \textit{websites} como podem acabar piorando o tempo de resposta para o usuário final.

No decorrer dos próximos anos o HTTP2 deve seguir o mesmo caminho do HTTP/1.1 e se tornar o protocolo mais utilizado da \textit{web}. Apesar de todo o esforço do HTTPbis (grupo responsável por desenvolver a especificação do HTTP2) em desenvolver um protocolo que garanta o melhor desempenho de \textit{websites} e aplicações, sempre é possível ser mais rápido se as medidas certas forem tomadas.

\section{Objetivos}
\label{sec:objetivos}

Este trabalho tem como objetivo analisar o comportamento de técnicas de otimização de desempenho de \textit{websites} desenvolvidas para os protocolos HTTP/1.0 e HTTP/1.1 quando aplicadas a \textit{websites} usando o protocolo HTTP2 e, se necessário, propor técnicas específicas para o novo protocolo.

Para realização do objetivo principal, os seguintes objetivos específicos foram determinados:

\begin{enumerate}
	\item Fazer uma análise comparativa das versões do protocolo HTTP
	\item Avaliar os ganhos de desempenho das técnicas propostas por Steve Souders ao aplicá-las ao HTTP2
	\item Se necessário, propor novas técnicas de otimização de desempenho de \textit{websites} específicas para o HTTP2
\end{enumerate}