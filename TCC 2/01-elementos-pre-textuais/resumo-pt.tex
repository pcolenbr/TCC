%
% Documento: Resumo (Português)
%

\begin{resumo}

Desde a década de 1990, a \textit{Web} não pára de se expandir e de adquirir novas funcionalidades. Nos dias atuais, a parte mais popular da Internet já não é mais composta de páginas estáticas que serviam apenas para leitura, mas de conteúdo gráfico e dinâmico. Juntamente com a evolução da \textit{Web}, a quantidade de dados que trafega entre servidores e usuários finais aumentou consideravelmente, isso graças à complexidade das páginas \textit{web}. Um aumento do tempo de carregamento das páginas resulta em usuários insatisfeitos e prejuízos para as empresas. Para diminuir esse tempo de carregamento, Steve Souders dedicou sua carreira em encontrar técnicas para otimização de desempenho de \textit{front-end} de \textit{websites}. Suas técnicas se provaram eficientes para o HTTP/1.1, que foi a versão vigente do protocolo HTTP por muitos anos. Mas em 2015 o HTTP/2 foi aprovado e está proto para começar a ser utilizado na \textit{Web}. Acredita-se que técnicas de otimização de desempenho que funcionavam para o HTTP/1.1, não terão o mesmo comportamento no HTTP/2. Para comprovar essa hipótese foi realizado um estudo de caso das técnicas propostas por Souders utilizando o servidor Nginx, versão 1.9.5. Os resultados encontrados foram inferiores aos esperados, mas este trabalho tenta explicar o motivo do tal ter acontecido através de novas hipóteses sobre o HTTP/2.

\end{resumo}