%
% Documento: Resumo (Português)
%

\begin{resumo}

Desde a década de 1990, a \textit{Web} não pára de se expandir e de adquirir novas funcionalidades. Nos dias atuais, a parte mais popular da Internet já não é mais composta de páginas estáticas que serviam apenas para leitura, mas de conteúdo gráfico e dinâmico, que interage com os usuários. Juntamente com a evolução da \textit{Web}, a quantidade de dados que trafega entre servidores e usuários finais aumentou consideravelmente, isso graças à complexidade das páginas \textit{web}. Um aumento do tempo de carregamento das páginas resulta em usuários insatisfeitos e prejuízos para empresas. Para diminuir esse tempo de carregamento, Steve Souders dedicou sua carreira em encontrar técnicas para otimização de desempenho de \textit{front-end} de \textit{websites}. Suas técnicas se provaram eficientes para o HTTP/1.1, que foi a versão vigente do protocolo HTTP por muitos anos. Com a iminência da liberação do HTTP/2, não se sabe se as técnicas de Souders continuarão funcionando para a nova versão do protocolo. Este trabalho analisou as técnicas propostas por Steve Souders, encontrou quais poderiam ser afetadas pelas mudanças do protocolo HTTP e as testou na nova versão. Como o protocolo HTTP/2 é muito novo, encontrar um servidor que suportasse a versão antiga e a mais recente do HTTP foi um desafio. Após algumas tentativas ficou definido que o Nginx, versão 1.9.5, seria usado como servidor de teste, apesar de ainda não ser uma versão oficial estável. Contudo, os testes com o HTTP/2 não comprovaram as hipóteses levantadas. O novo protocolo acabou obtendo resultados de desempenho piores do que sua versão prévia. Acredita-se que esse resultado negativo é consequência de falhas no suporte ao paralelismo no Nginx ou mesmo no computador de teste, ou ainda a falta de adaptação dos navegadores \textit{web} para atender todas as funcionalidades do HTTP/2

\end{resumo}