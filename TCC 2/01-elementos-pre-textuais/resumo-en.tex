%
% Documento: Resumo (Inglês)
%

\begin{resumo}[Abstract]

Since the 90s the Web has not stoped expanding and acquiring new features. Nowadays the most popular part of the Internet is not made of static pages used just for reading, it is made of graphical and dynamic content that interacts with the final user. Along with the evolution of the Web, the amount of data exchanged between servers and final users has increased considerably, and that is because of the complexity of the web pages. An increase in pages' load time results in dissatisfied users and financial losses for companies. To reduce this wait time, Steve Souders dedicated his career to find techniques to optimize the performance of the front-end of websites. His techniques have been proved efficient for HTTP/1.1, that was the ruling version of the protocol for a long day. But in 2015 the HTTP/2 was approved and it is ready to be used by the \textit{Web}. It is believed that the performance optimization techniques that used to work on HTTP/1.1 will not have the same behavior on the HTTP/2. To prove this hypothesis it was held a case study to analyse Souders' techniques using the Nginx server, version 1.9.5. The results found were inferior than the results expected, but this work tries to explain the reason for this too happen proposing new hypothesis about the HTTP/2.

\end{resumo}
