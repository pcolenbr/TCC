%
% Documento: Resumo (Inglês)
%

\begin{resumo}[Abstract]

Since the 90s the Web has not stoped expanding and acquiring new features. Nowadays the most popular part of the Internet is not made of static pages used just for reading, it is made of graphical and dynamic content that interacts with the final user.. Along with the evolution of the Web, the amount of data exchanged between servers and final users has increased considerably, and that is because of the complexity of the web pages. An increase in pages' load time results in dissatisfied users and financial losses for companies. To reduce this wait time, Steve Souders dedicated his career to find techniques to optimize the performance of the front-end of websites. His techniques have been proved efficient for HTTP/1.1, that was the ruling version of the protocol for a long day. With the imminent release of HTTP/2, the techniques created by Souders might not work in the new protocol. This work analyzed the techniques proposed by Steve Souders, found the ones that could be affected by the changes in the HTTP protocol and tested them in the new version. Because the HTTP/2 protocol is new, finding a server that supports both versions of HTTP was a challenge. After some tries it was defined that the Nninx, version 1.9.5, would be used as the test serve, even though it is not a offical version. However, that tests with the HTTP/2 did not prove the hypotheses. The new protocol had results worse than its previouos versions. It is believed that this negative results is a consequence of the lack of support for parallelism on the Nginx implementation or even in the test computer, or yet it might be lack of development of the web browsers to support all HTTP/2 features.

\end{resumo}
