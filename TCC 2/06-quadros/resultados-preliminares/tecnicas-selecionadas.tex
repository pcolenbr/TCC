\begin{quadro}[!htb]
	\centering
	\caption{Técnicas selecionadas.\label{qua:tecnicasselecionadas}}
\end{quadro}
\begin{tabularx}{\textwidth}{| X | X |}
	\hline
	\multicolumn{2}{| c |}{\cellcolor[HTML]{C0C0C0}\textbf{Técnicas que serão testadas}} \\
	\hline
	\textbf{Técnica} & \textbf{Motivo} \\
	\hline
	Faça menos requisições HTTP & A multiplexação mudará a velocidade de carregamento das requisições paralelas. \\
	\hline
	Reduza o número de pesquisas de DNS & A pesquisa por DNSs que não estão em \textit{cache} será alterada com a nova versão do protocolo HTTP. \\
	\hline
	Lidando com \textit{scripts} assíncronos & Com as funcionalidades de dependência e prioridade do HTTP/2, \textit{scripts} assíncronos não serão mais um problema e essa técnica tende a cair em desuso. \\
	\hline
	Evite redirecionamentos & Com o HTTP/2 espera-se que o processo de redirecionamento seja mais rápido e o desempenho dos redirecionamentos melhor. \\
	\hline
	Quebrando domínios dominantes & Domínios dominantes diminuem a possibilidade de paralelismo do HTTP. Mas no HTTP/2 a maneira de lidar com o carregamento de componentes que estão no mesmo domínio será alterada graças à multiplexação. \\
	\hline
	Entregando o documento cedo & Dividir a carga inicial possibilita que as páginas comecem a ser carregadas mais cedo e isso diminui o tempo total de carregamento. Apesar de em primeira análise não parecer que essa técnica sofrerá alterações, isso vai depender de como o HTTP/2 lida com as funções de descarga das linguagens de \textit{back-end}. \\
	\hline
	\multicolumn{2}{| c |}{\cellcolor[HTML]{C0C0C0}\textbf{Técnicas que não serão testadas}} \\
	\hline
	\textbf{Técnica} & \textbf{Motivo} \\
	\hline
	Use Redes de Entrega de Conteúdo (CDN) & Diminuir a distância física entre dois pontos sempre acarretará em um tempo mais curto de resposta. \\
	\hline
	Adicione cabeçalhos de expiração & Utilizar cabeçalhos de expiração melhora o funcionamento da \textit{cache} e, consequentemente, diminui o tempo de carregamento das páginas \textit{web}. \\
	\hline
	Utilize \textit{gzip} no componentes & O HPACK que será o novo modelo de compressão de dados do protocolo. \\
	\hline
	Coloque folhas de estilo no topo da página & O objetivo principal dessa técnica é melhorar a responsividade das páginas \textit{web} e não diminuir o seu tempo de carregamento. \\
	\hline
	Coloque \textit{scripts} no fim da página & Essa técnica não tem como objetivo reduzir do tempo de carregamento das páginas. \\
	\hline
	Evite expressões CSS & Expressões CSS não são mais utilizadas pelos desenvolvedores. \\
	\hline
	Faça arquivos \textit{JavaScripts} e folhas de estilo externos & Arquivos \textit{JavaScripts} e folhas de estilo externos podem ser salvos em \textit{cache} e isso sempre melhorará o desempenho da páginas. \\
	\hline
	Minimizar arquivo \textit{JavaScript} & Respostas menores sempre serão vantajosas para o desempenho de uma página. \\
	\hline
	Remova \textit{scripts} duplicados & \textit{Scripts} duplicados aumentam o tamanho dos arquivos e isso faz com que as requisições HTTP sejam mais lentas. \\
	\hline
	Configure \textit{ETags} & Como não são utilizadas em larga escala não são relevantes ao escopo deste trabalho. \\
	\hline
	Habilite \textit{cache} para AJAX & A \textit{cache} melhora o desempenho de páginas \textit{web} independente da versão do protocolo HTTP. \\
	\hline
	Dividindo carga inicial & O objetivo de dividir a carga inicial é melhorar a responsividade da página, mas essa técnica não afeta o tempo total de carregamento. \\
	\hline
	Carregando \textit{scripts} sem bloqueios & A característica de bloqueio no carregamento de \textit{scripts} é definida pelo HTML5 e não pelo protocolo HTTP. \\
	\hline
	Posicionando blocos de \textit{scripts} em linha & O posicionamento de \textit{scripts} em linha é importante para evitar que as páginas fiquem bloqueadas para o usuário. Essa característica é determinada pelo HTML5 e por isso não terá alteração no HTTP/2. \\
	\hline
	Escrevendo \textit{JavaScripts} eficientes & Está fora do escopo deste trabalho analisar a eficiência de \textit{scripts}. \\
	\hline
	Escalando usando \textit{Comet} & O \textit{Comet} melhora o desempenho de aplicações com um número muito grande de chamadas AJAX, mas está fora do escopo deste trabalho analisar o funcionamento dessa técnica. \\
	\hline
	Otimizando imagens & Otimizar imagens tem como objetivo reduzir o tamanho desses arquivos. Como requisições menores sempre apresentarão desempenho melhor. \\
	\hline
	Usando \textit{Iframes} com moderação & Continua sendo aconselhado não utilizar \textit{Iframes} pois eles continuarão afetando \textit{websites} negativamente. \\
	\hline
	Simplificando seletor CSS & Apesar de diminuírem o tempo de carregamento da página (em escala muito pequena), a pesquisa de seletores CSS é definida pelos navegadores e não pelo protocolo HTTP. \\
	\hline
\end{tabularx}