%
% Documento: Resultados
%

\chapter{Resultados}

Após a execução dos testes, foram geradas tabelas com os valores dos tempos de carregamento, desvio padrão dos tempos para cada técnica e gráficos em cascata das páginas \textit{web} desenvolvidas para cada técnica de otimização analisada. Os valores registrados mostraram as diferenças no comportamento das versões do protocolo HTTP e como cada técnica se comportaria no HTTP/2.

Neste capítulo são descritos e analisados os valores encontrados e é analisado o comportamento do desempenho das páginas nas duas versões do protocolo.

\section{Análise do tempo de carregamento}
\label{analisedotempodecarregamento}

O tempo de carregamento foi a métrica de desempenho escolhida para comparar o comportamento das páginas de teste nos diferentes protocolos. Essa métrica foi obtida com a ajuda da ferramenta de desenvolvedor de cada navegador. As tabelas \ref{resultados-facamenosrequisicoeshttp} a \ref{resultados-template}, mostram os resultados obtidos para as médias, desvio padrão e coeficiente de variação do tempo carregamento das páginas \textit{web} divididos por técnica analisada. Os testes foram executados 25 vezes para cada cada cenário e as medidas estatísticas foram calculadas a partir dos valores de tempo de carregamento obtidos em cada execução do teste. Os resultados detalhados para cada técnica, mostrando os valores para cada execução, podem ser encontrados no \autoref{apend:tabelasdetalhadas}.

\begin{table}[H]
\centering
\caption{Resultados da técnica ''Faça menos requisições HTTP''}
\label{resultados-facamenosrequisicoeshttp}
\begin{tabular}{ll|c|c|c|c|}
\cline{3-6}
                                                                                                                & \multicolumn{1}{c|}{\textbf{}}   & \multicolumn{2}{c|}{\textbf{HTTP/1.1}} & \multicolumn{2}{c|}{\textbf{HTTP/2}} \\ \cline{3-6} 
\multicolumn{1}{c}{}                                                                                            &                                  & \textbf{Firefox}   & \textbf{Chrome}   & \textbf{Firefox}  & \textbf{Chrome}  \\ \hline
\multicolumn{1}{|l|}{\multirow{3}{*}{\textbf{\begin{tabular}[c]{@{}l@{}}Arquivos\\ separados\end{tabular}}}}    & \textbf{Média}                   & 4.8996             & 4.944            & 4.8944             & 4.8184           \\ \cline{2-6} 
\multicolumn{1}{|l|}{}                                                                                          & \textbf{Desvio padrão}           & 0.15053            & 0.11193          & 0.10828           & 0.09029          \\ \cline{2-6} 
\multicolumn{1}{|l|}{}                                                                                          & \textbf{Coeficiente de varição}  & 3.24\%             & 2.43\%            & 2.33\%            & 1.97\%           \\ \hline
\multicolumn{1}{|l|}{\multirow{3}{*}{\textbf{\begin{tabular}[c]{@{}l@{}}Arquivos\\ concatenados\end{tabular}}}} & \textbf{Média}                   & 4.6496             & 4.6528             & 4.608            & 4.5844           \\ \cline{2-6} 
\multicolumn{1}{|l|}{}                                                                                          & \textbf{Desvio padrão}           & 0.16059            & 0.20940           & 0.11883           & 0.07786          \\ \cline{2-6} 
\multicolumn{1}{|l|}{}                                                                                          & \textbf{Coeficiente de variação} & 3.28\%             & 4.28\%            & 2.40\%            & 1.62\%           \\ \hline
\end{tabular}
\end{table}

A \autoref{resultados-facamenosrequisicoeshttp} mostra os resultados obtidos com a redução de requisições HTTP concatenando arquivos CSS e JS. Diferente do esperado, o HTTP/2 obteve resultados piores ou iguais ao HTTP/1.1 e ainda a concatenação dos arquivos gerou uma melhora no tempo de carregamento da página, o que não era esperado para o novo protocolo.

\begin{table}[H]
\centering
\caption{Resultados da técnica ''Reduza o número de pesquisas DNS''}
\label{resultados-reduzaonumerodepesquisasdns}
\begin{tabular}{ll|c|c|c|c|}
\cline{3-6}
                                                                                                         & \multicolumn{1}{c|}{\textbf{}}   & \multicolumn{2}{c|}{\textbf{HTTP/1.1}} & \multicolumn{2}{c|}{\textbf{HTTP/2}} \\ \cline{3-6} 
\multicolumn{1}{c}{}                                                                                     &                                  & \textbf{Firefox}   & \textbf{Chrome}   & \textbf{Firefox}  & \textbf{Chrome}  \\ \hline
\multicolumn{1}{|l|}{\multirow{3}{*}{\textbf{\begin{tabular}[c]{@{}l@{}}Multiplas\\ CDNs\end{tabular}}}} & \textbf{Média}                   & 4.9                & 4.2304            & 4.3484            & 4.1972           \\ \cline{2-6} 
\multicolumn{1}{|l|}{}                                                                                   & \textbf{Desvio padrão}           & 0.12958            & 0.13001           & 0.11417           & 0.09220          \\ \cline{2-6} 
\multicolumn{1}{|l|}{}                                                                                   & \textbf{Coeficiente de varição}  & 2.65\%             & 2.70\%            & 3.01\%            & 2.20\%           \\ \hline
\multicolumn{1}{|l|}{\multirow{3}{*}{\textbf{\begin{tabular}[c]{@{}l@{}}Única\\ CDN\end{tabular}}}}      & \textbf{Média}                   & 5.0664             & 4.2104            & 4.314             & 4.0712           \\ \cline{2-6} 
\multicolumn{1}{|l|}{}                                                                                   & \textbf{Desvio padrão}           & 0.26658            & 0.17080           & 0.23863           & 0.04958          \\ \cline{2-6} 
\multicolumn{1}{|l|}{}                                                                                   & \textbf{Coeficiente de variação} & 5.26\%             & 5.53\%            & 4.06\%            & 1.22\%           \\ \hline
\end{tabular}
\end{table}

Analisando a \autoref{resultados-reduzaonumerodepesquisasdns} percebe-se que o HTTP/2, apesar de obter desempenho melhor do que o HTTP/1.1, teve um aumento no tempo de carregamento para um número maior de CDNs. Esse aumento pode ser consequência do tempo de busca de DNS ter aumentado ou por que as novas CDNs utilizadas possuíam desempenho ruim. Mas, julgando pelo fato de todas serem CDNs grandes e muito utilizadas, considera-se que todas são otimizadas e possuem alto desempenho. 

\begin{table}[H]
\centering
\caption{Resultados da técnica ''Quebrando domínios dominantes''}
\label{resultados-quebrandodominiosdominantes}
\begin{tabular}{ll|c|c|c|c|}
\cline{3-6}
                                                        &                                 & \multicolumn{2}{c|}{\textbf{HTTP/1.1}} & \multicolumn{2}{c|}{\textbf{HTTP/2}} \\ \cline{3-6} 
                                                        &                                 & \textbf{Firefox}  & \textbf{Chrome}    & \textbf{Firefox}  & \textbf{Chrome}  \\ \hline
\multicolumn{1}{|l|}{}                                  & \textbf{Média}                  & 5.272             & 4.1912             & 4.5648            & 4.2              \\ \cline{2-6} 
\multicolumn{1}{|l|}{}                                  & \textbf{Desvio padrão}          & 0.37758           & 0.19079            & 0.35197           & 0.27013          \\ \cline{2-6} 
\multicolumn{1}{|l|}{\multirow{-3}{*}{\textbf{2 CDNs}}} & \textbf{Coeficiente de varição} & 7.16\%            & 4.55\%             & 7.71\%            & 6.43\%           \\ \hline
\multicolumn{1}{|l|}{}                                  & \textbf{Média}                  & 5.2572            & 4.4652             & 5.2648            & 4.2872           \\ \cline{2-6} 
\multicolumn{1}{|l|}{}                                  & \textbf{Desvio padrão}          & 0.27470           & 0.35509            & 0.48068           & 0.26890          \\ \cline{2-6} 
\multicolumn{1}{|l|}{\multirow{-3}{*}{\textbf{3 CDNs}}} & \textbf{Coeficiente de varição} & 5.23\%            & 9.13\%             & 7.95\%            & 6.27\%           \\ \hline
\rowcolor[HTML]{EFEFEF} 
\multicolumn{2}{|c|}{\cellcolor[HTML]{EFEFEF}\textbf{Porcentual de Melhora (Média):}}     & \textbf{0.28\%}   & \textbf{-15.33\%}  & \textbf{-6.54\%}  & \textbf{-2.08\%} \\ \hline
\end{tabular}
\end{table}

A \autoref{resultados-quebrandodominiosdominantes} mostra os resultados obtidos para o tempo de carregamento da página utilizando duas e três CDNs respectivamente. Mais uma vez, percebe-se que o HTTP/2 possui desempenho melhor do que o HTTP/1.1, e além disso que o número de CDNs usado interfere diretamente no tempo de carregamento da página, apesar de o resultado esperado pela análise da especificação do novo protocolo sugerir que o número de CDNs não fosse interferir no tempo de carregamento das páginas \textit{web}.

Outro motivo para isso pode ser o fato de as CDNs não suportarem requisições HTTP/2, sendo assim, a comunicação com as CDNs ocorre através da versão antiga do protocolo, fato que acaba influenciando no ganho de desempenho que a página teria com o HTTP/2.

\begin{table}[H]
\centering
\caption{Resultados da técnica ''Evite redirecionamento''}
\label{resultados-redirecionamento}
\begin{tabular}{ll|c|c|c|c|}
\cline{3-6}
                                                                                                                 & \multicolumn{1}{c|}{\textbf{}}  & \multicolumn{2}{c|}{\textbf{HTTP/1.1}} & \multicolumn{2}{c|}{\textbf{HTTP/2}} \\ \cline{3-6} 
\multicolumn{1}{c}{}                                                                                             &                                 & \textbf{Firefox}   & \textbf{Chrome}   & \textbf{Firefox}  & \textbf{Chrome}  \\ \hline
\multicolumn{1}{|l|}{\multirow{3}{*}{\textbf{\begin{tabular}[c]{@{}l@{}}Evite\\ Redirecionamento\end{tabular}}}} & \textbf{Média}                  & 6.5684             & 6.7612            & 7.022             & 7.2168           \\ \cline{2-6} 
\multicolumn{1}{|l|}{}                                                                                           & \textbf{Desvio padrão}          & 1.01423            & 0.86158           & 0.51653           & 0.76464          \\ \cline{2-6} 
\multicolumn{1}{|l|}{}                                                                                           & \textbf{Coeficiente de varição} & 15.44\%            & 7.36\%            & 12.74\%           & 10.60\%          \\ \hline
\end{tabular}
\end{table}

Apesar de ser esperado que o redirecionamento não influencie no tempo de carregamento da página no HTTP/2 quando as tabelas \ref{resultados-facamenosrequisicoeshttp} e \ref{resultados-redirecionamento} são comparadas, perceber-se que, quando a página é redirecionada, o desempenho da mesma é pior independente do protocolo utilizado.

\begin{table}[h]
	\centering
	\caption{Resultados do teste final utilizando \textit{template}.}
	\label{resultados-template}
	\begin{tabular}{cccc}
		\hline
		\multicolumn{2}{c}{\textbf{Firefox}} & \multicolumn{2}{c}{\textbf{Chrome}} \\
		\hline
		HTTP/1.1 & HTTP/2 & HTTP/1.1 & HTTP/2 \\
		\hline
		0.93 & 0.95 & 1.2 & 0.95 \\
		0.96 & 1.13 & 0.96 & 1.12 \\
		0.98 & 0.97 & 1.1 & 0.97 \\
		1.01 & 1 & 0.98 & 1.06 \\
		1   & 0.99 & 0.98 & 0.99 \\
		1.02 & 1.06 & 0.96 & 0.94 \\
		0.98 & 0.94 & 0.91 & 0.95 \\
		1.07 & 0.89 & 0.94 & 0.94 \\
		0.98 & 1.01 & 1.14 & 0.97 \\
		0.93 & 0.95 & 0.91 & 0.98 \\
		0.9 & 1 & 0.95 & 0.97 \\
		0.99 & 0.99 & 0.94 & 0.9  \\
		0.93 & 1.02 & 0.91 & 0.94 \\
		0.91 & 1.08 & 0.99 & 0.97 \\
		0.99 & 0.98 & 0.95 & 1    \\
		0.92 & 0.92 & 1.02 & 0.96 \\
		1.12 & 1.08 & 0.95 & 1.03 \\
		0.9 & 0.98 & 0.97 & 0.96 \\
		0.94 & 1.09 & 1.2 & 1    \\
		0.93 & 1.02 & 0.92 & 0.95 \\
		0.9 & 1.07 & 0.99 & 0.92 \\
		0.91 & 0.92 & 0.94 & 1.05 \\
		0.93 & 0.89 & 0.9 & 0.97 \\
		0.94 & 0.98 & 1.06 & 1.01 \\
		0.97 & 1.03 & 0.89 & 1.03 \\
		\hline
		\multicolumn{4}{c}{\textbf{Média}} \\
		0.9616 & 0.9976 & 0.9864 & 0.9812 \\
		\hline
	\end{tabular}
\end{table}

Na \autoref{resultados-template} encontra-se o resultado do tempo de carregamento da página utilizando o \textit{template} escolhido. Pode-se dizer que os dois protocolos tiveram o mesmo resultado, pois as diferenças são muito pequenas. Contudo, era esperado que o HTTP/2 fosse mais eficiente e carregasse a página mais rapidamente.

\subsection{Análise do desvio padrão}
\label{desviopadrao}

Apesar do HTTP/2 não apresentar desempenho melhor do que HTTP/1.1 para todos os cenários e de algumas técnicas que não deveriam alterar o comportamento das páginas no protocolo causarem um efeito não esperado, pode-se notar que o novo protocolo se comporta diferente de sua versão prévia.

Observando os desvios padrão mostrados nas tabelas \ref{resultados-facamenosrequisicoeshttp} à \ref{resultados-template} percebe-se que o HTTP/2 é um protocolo mais estável e que, aparentemente, sofre menos com as alterações de rede. Quando utilizando uma conexão de rede real, não é possível remover 100\% os picos e vales de conexão que ocorrem. Por esse motivo percebemos que os tempos de carregamento se alteram (apesar de os valores muito fora da média já terem sido retirados). Mesmo assim, o desvio padrão do HTTP/2 foi menor do que o do HTTP/1.1 para todas as páginas, o que pode ser uma grande vantagem para desenvolvedores tentarem prever o comportamento de seus \textit{websites} e seus usuários.

\subsection{Cascata}
\label{cascata}

As figuras \ref{fig:cascatahttp11} e \ref{fig:cascatahttp2} mostram o carregamento do cenário descrito na \autoref{facamenosrequisicoeshttp} em formato de gráfico em cascata, obtido com a ajuda da ferramenta de desenvolvedor do Google Chrome.

\begin{landscape}
	\begin{figure}[!htbp]
    	\centering
	    \caption{Cascata representando \textit{download} de componentes da página no HTTP/1.1.}
    	\includegraphics[width=1.5\textwidth]{./04-figuras/analise-de-resultados/cascata_http11}
	    \label{fig:cascatahttp11}
	\end{figure}
\end{landscape}

\begin{landscape}
	\begin{figure}[!htbp]
    	\centering
	    \caption{Cascata representando \textit{download} de componentes da página no HTTP/2.}
    	\includegraphics[width=1.5\textwidth]{./04-figuras/analise-de-resultados/cascata_http2}
	    \label{fig:cascatahttp2}
	\end{figure}
\end{landscape}

Nas Figuras \ref{fig:cascatahttp11} e \ref{fig:cascatahttp2}, a cor marrom representa o tempo que o componente ficou bloqueado, ou seja, estava aguardando outros componentes serem baixados para poder começar a ser processado. Sendo assim, durante esse período, nem mesmo a requisição pelo componente foi feita ainda. A porção representada pela cor roxa é o tempo que o cliente ficou esperando pelo servidor para processar sua requisição e responde-la. E a cor cinza representa o tempo de envio do componente, ou seja, o tempo que o cliente demorou para fazer o \textit{download}.

Os gráficos de cascata para os outros cenários seguem o mesmo padrão dos dois mostrados e podem ser encontrados no \autoref{apend:graficosemcascata}. Quando analisadas, essas figuras revelam a diferença no comportamento dos protocolos HTTP/1.1 e HTTP/2. Percebe-se que no HTTP/1.1 os gráficos possuem o formato conhecido como cascata, em que os componentes ficam bloqueados ou não são nem processados até que outros já tenham terminado de serem baixados pelo cliente. Enquanto que no HTTP/2, a cascata não é formada, isso graças ao grande paralelismo que esse protocolo possui.

Com isso, esperava-se que o tempo gasto com componentes bloqueados no HTTP/1.1 fosse removido do tempo total de carregamento no HTTP/2. Mas podemos perceber, analisando os resultados do tempo de carregamento final das páginas, que isso não acontece. Apesar de os componentes não ficarem bloqueados, eles possuem tempo de \textit{download} muito maiores no HTTP/2.

Existem alguns fatores que podem ser a causa desse aumento no tempo de \textit{download} das páginas. Tsujikawa, desenvolvedor do \textit{ngtthp2}, acredita que um dos motivos é o fato de os navegadores não estarem preparados para utilizar todas as funcionalidades disponíveis no HTTP/2, afinal de contas o protocolo é muito novo e ainda é necessário tempo para que tal adaptação seja feita\footnote{https://github.com/tatsuhiro-t/nghttp2/issues/387}. Outros motivos que podem estar limitando os ganhos com o paralelismo do HTTP/2 são a possível falta de otimização do Nginx, que não conseguiria servir tantas requisições paralelas otimizadas ao mesmo tempo, a limitação do \textit{hardware} da máquina onde está instalado o servidor, que não possui um processador multi-tarefas, ou até mesmo limitações da rede, que possui largura de banda para transferir tantos dados ao mesmo tempo.

\section{Teste T}
\label{testet}

Para se comprovar o nível de confiança dos resultados encontrados, foi realizada uma análise estatística dos testes. Essa análise utilizou o Teste T para entender se os ganhos e perdas no tempo de carregamento das páginas ocorreu por acaso ou se os valores encontrados são aceitos estatisticamente.

As Tabelas \ref{testt-facamenosrequisicoeshttp} a \ref{testt-eviteredirecionamento} mostram os valores encontrados para P e qual conclusão pode ser tirada com a análise desse valor para cada técnica. Para essa análise foi considerado um nível de confiança de 95\%, ou seja, se o valor de P for menor do que 5\% os resultados do teste são válidos estatisticamente e não ocorreram por acaso. Caso o contrário, o teste não é conclusivo, pois há 95\% de chance da diferença obtida com a aplicação da técnica ter ocorrido devido o acaso e nesse caso os testes deverão ser repetidos com um número maior de amostras.

\begin{table}[H]
\centering
\caption{Valores de P: ''Faça menos requisições HTTP''}
\label{testt-facamenosrequisicoeshttp}
\begin{tabular}{|c|c|c|c|}
\hline
\multicolumn{2}{|c|}{\textbf{Firefox}}                                                                                                            & \multicolumn{2}{c|}{\textbf{Chrome}}                                                                                                              \\ \hline
\textbf{HTTP/1.1}                                                       & \textbf{HTTP/2}                                                         & \textbf{HTTP/1.1}                                                       & \textbf{HTTP/2}                                                         \\ \hline
0.0001166\%                                                             & 0.0000000\%                                                             & 0.0000859573\%                                                          & 0.0000000001\%                                                          \\ \hline
\multicolumn{4}{|c|}{\textbf{Conclusão}}                                                                                                                                                                                                                                                              \\ \hline
\begin{tabular}[c]{@{}c@{}}Técnica melhorou\\ o desempenho\end{tabular} & \begin{tabular}[c]{@{}c@{}}Técnica melhorou\\ o desempenho\end{tabular} & \begin{tabular}[c]{@{}c@{}}Técnica melhorou\\ o desempenho\end{tabular} & \begin{tabular}[c]{@{}c@{}}Técnica melhorou\\ o desempenho\end{tabular} \\ \hline
\end{tabular}
\end{table}
\begin{table}[H]
\centering
\caption{Valores de P: ''Reduza o número de pesquisas DNS''}
\label{testt-reduzaonumerodepesquisasdns}
\begin{tabular}{|c|c|c|c|}
\hline
\multicolumn{2}{|c|}{\textbf{Firefox}}                                                                                                         & \multicolumn{2}{c|}{\textbf{Chrome}}                                                                                                           \\ \hline
\textbf{HTTP/1.1}                                                       & \textbf{HTTP/2}                                                      & \textbf{HTTP/1.1}                                                    & \textbf{HTTP/2}                                                         \\ \hline
0.7952776\%                                                             & 86.3540918\%                                                         & 63.5902811929\%                                                      & 0.0000882758\%                                                          \\ \hline
\multicolumn{4}{|c|}{\textbf{Conclusão}}                                                                                                                                                                                                                                                        \\ \hline
\begin{tabular}[c]{@{}c@{}}Técnica melhorou\\ o desempenho\end{tabular} & \begin{tabular}[c]{@{}c@{}}Sem variação\\ significativa\end{tabular} & \begin{tabular}[c]{@{}c@{}}Sem variação\\ significativa\end{tabular} & \begin{tabular}[c]{@{}c@{}}Técnica melhorou\\ o desempenho\end{tabular} \\ \hline
\end{tabular}
\end{table}
\begin{table}[H]
\centering
\caption{Valores de P: ''Quebrando domínios dominantes''}
\label{testt-quebrandodominiosdominantes}
\begin{tabular}{|c|c|c|c|}
\hline
\multicolumn{2}{|c|}{\textbf{Firefox}}                                                                                                         & \multicolumn{2}{c|}{\textbf{Chrome}}                                                                                                           \\ \hline
\textbf{HTTP/1.1}                                                    & \textbf{HTTP/2}                                                         & \textbf{HTTP/1.1}                                                       & \textbf{HTTP/2}                                                      \\ \hline
87.7314494\%                                                         & 0.0000770\%                                                             & 0.1984000460\%                                                          & 26.7952938515\%                                                      \\ \hline
\multicolumn{4}{|c|}{\textbf{Conclusão}}                                                                                                                                                                                                                                                        \\ \hline
\begin{tabular}[c]{@{}c@{}}Sem variação\\ significativa\end{tabular} & \begin{tabular}[c]{@{}c@{}}Técnica melhorou\\ o desempenho\end{tabular} & \begin{tabular}[c]{@{}c@{}}Técnica melhorou\\ o desempenho\end{tabular} & \begin{tabular}[c]{@{}c@{}}Sem variação\\ significativa\end{tabular} \\ \hline
\end{tabular}
\end{table}
\begin{table}[H]
\centering
\caption{Valores de P: ''Evite redirecionamento''}
\label{testt-eviteredirecionamento}
\begin{tabular}{|c|c|c|c|}
\hline
\multicolumn{2}{|c|}{\textbf{Firefox}}                                                                                                            & \multicolumn{2}{c|}{\textbf{Chrome}}                                                                                                              \\ \hline
\textbf{HTTP/1.1}                                                       & \textbf{HTTP/2}                                                         & \textbf{HTTP/1.1}                                                       & \textbf{HTTP/2}                                                         \\ \hline
0.0000024\%                                                             & 0.0000000\%                                                             & 0.0000000078\%                                                          & 0.0000000000\%                                                          \\ \hline
\multicolumn{4}{|c|}{\textbf{Conclusão}}                                                                                                                                                                                                                                                              \\ \hline
\begin{tabular}[c]{@{}c@{}}Técnica melhorou\\ o desempenho\end{tabular} & \begin{tabular}[c]{@{}c@{}}Técnica melhorou\\ o desempenho\end{tabular} & \begin{tabular}[c]{@{}c@{}}Técnica melhorou\\ o desempenho\end{tabular} & \begin{tabular}[c]{@{}c@{}}Técnica melhorou\\ o desempenho\end{tabular} \\ \hline
\end{tabular}
\end{table}

\section{Discussão dos resultados}
\label{discussaodosresultados}

A hipótese inicial deste trabalho propunha que o HTTP/2 apresentaria um desempenho melhor do que o HTTP/1.1 e que algumas técnicas de otimização de desempenho utilizadas para o protocolo mais antigo não teriam efeito ou seriam até mesmo prejudiciais para as páginas \textit{web} no novo protocolo. Mas apesar do esforço para encontrar um servidor HTTP/2 confiável e se comprovar a hipótese proposta, o HTTP/2 não apresentou o comportamento esperado.

A redução de requisições HTTP continuou tendo aumentando o desempenho das páginas \textit{web}, o aumento de pesquisas de DNS foi prejudicial e o redirecionamento ainda deve ser um recurso evitado. Além disso, quando analisado em uma simulação de um site real sem sobre carga de requisições e sem a aplicação de técnicas de otimização, o HTTP/2 ainda obteve resultados piores do que o HTTP/1.1.

Outro ponto que vale ser observado são os testes que não foram validados pelo Teste T. Existem 4 (quatro) situações onde os testes não podem ser conclusivos pois os resultados encontrados podem te ocorrido devido ao acaso. Nesses casos os testes deverão ser repetidos com um número maior de amostras para se entender o que realmente acontece quando a técnica de otimização em questão é aplicada.

Não foi possível analisar a fundo o motivo de porquê o novo protocolo não se comportou como o esperado, mas algumas perguntas ficam em aberto para serem respondidas em trabalhos futuros.

O HTTP/2 ainda está em fase de desenvolvimento e adaptação e, no decorrer dos próximos meses muitas implementações devem ser lançadas para que sejam testadas pelos usuários da \textit{Web}. Apesar de não ter obtido resultado inicialmente esperado neste trabalho, acredita-se ainda que o novo protocolo vai ser uma grande mudança para a \textit{Web} e que as técnicas de otimização de \textit{front-end} de \textit{websites} deverão se atualizar para serem efetivas no novo protocolo.

\section{Questões em aberto}
\label{questoesemaberto}

A seguir, encontra-se uma lista de questões deixadas em aberto por este trabalho que podem ser usadas como guias de trabalhos futuros.

\begin{itemize}
	\item Porque a implementação atual do HTTP/2 apresentou desempenho inferior do que o HTTP/1.1?
	\item Quanto o \textit{hardware} da máquina onde está instalado o servidor \textit{web} interfere no desempenho do HTTP/2?
	\item O Nginx, versão 1.9.5, consegue lidar com uma grande quantidade de requisições paralelas?
	\item Qual é o número de componentes ideal em uma página, de forma que o paralelismo não gera perda de desempenho?
	\item Os navegadores \textit{web} estão preparados para utilizar todo o potencial do HTTP/2?
	\item Porque reduzir as requisições HTTP melhorou o desempenho da página no HTTP/2?
\end{itemize}

Espera-se que as respostas para tais questões ajudem a entender o motivo dos testes deste trabalho não obterem os resultados esperados.