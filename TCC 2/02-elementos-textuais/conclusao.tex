%
% Documento: Conclusão
%

\chapter{Conclusão}

A \textit{Web} é a aplicação mais utilizada na rede mundial de computadores, a Internet, e grande parte das trocas de informações na \textit{Web} ocorrem utilizando o protocolo HTTP. O protocolo HTTP define cabeçalhos para requisições e respostas e com esses cabeçalhos consegue formalizar como será feita a transferência de dados entre clientes e servidores. Após quase 17 anos o protocolo finalmente recebe uma atualização chamada de HTTP/2, que já foi aprovada e está sendo adotada aos poucos.

Dentro da \textit{Web} os \textit{websites} são ferramentas muito importantes para a economia global e o dia a dia de bilhões de pessoas. A medida que a \textit{Web} foi evoluindo os \textit{websites} deixaram de ser páginas estáticas com conteúdo textual e passaram a interagir com os usuários, passaram a se movimentar e a depender de imagens e vídeos para transmitir as ideias desejadas por seus proprietários. O problema do aumento de elementos visuais nas páginas \textit{web} é que eles aumentam a quantidade de dados que trafegam na rede, e mesmo com a melhora na velocidade das conexões, maior tem que ser o esforço para garantir a qualidade do serviço. Durante muitos anos todo o tempo dedicado à otimização de desempenho de \textit{websites} focou-se no \textit{back-end}, até que Steve Souders mostrou que 80-90\% do tempo de carregamento das páginas era gasto no \textit{front-end}. Por esse motivo Souders propôs várias técnicas que poderiam ser utilizadas para tornar as páginas \textit{web} mais rápidas e comprovou a eficacia dessas técnicas no HTTP/1.1.

Este trabalho tentou analisar como as técnicas propostas por Steve Souders se comportariam quando aplicadas em páginas \textit{web} que utilizassem o protocolo HTTP/2. Como pré-requisito à análise das técnicas, foi feita uma comparação entre as versões do protocolo HTTP e antes de começar qualquer configuração ou implementação, foi necessário definir quais técnicas poderiam sofrer alterações com a mudança de versão do protocolo. Sabendo quais técnicas seriam testadas foi feita uma busca por implementações gratuitas do novo protocolo e a configuração de um servidor com suporte aos HTTP/1.1 e HTTP/2. Com o servidor funcionando bastou comparar o tempo de carregamento das páginas nas diferentes versões.

Como o protocolo HTTP/2 é muito recente foi difícil encontrar um único servidor que tivesse suporte à ele e ao HTTP/1.1. Então, após várias tentativas falhas com outras implementações, ficou definido que o \textit{nghttp2} seria a implementação escolhida para se executar o HTTP/2. Já para a execução do HTTP/1.1, a decisão ficou entre os servidores Apache e Nginx. Por apresentar melhor desempenho de carregamento de páginas o Nginx foi o escolhido.

Os tempos de carregamento registrados com a execução dos testes nesses dois servidores não foram satisfatórios. Levando em conta que o protocolo HTTP/2 está avançando mais a cada dia, foi feita uma nova busca por servidores com suporte a versão mais recente do HTTP. Encontrou-se uma versão recente e não-estável do Nginx, 1.9.5, com suporte às duas versões do protocolo, e ficou decidido que os testes seriam repetidos utilizando agora um mesmo servidor para os HTTP/1.1 e HTTP/2.

Apesar dos testes terem sido executados no mesmo servidor, os resultados encontrados continuaram sendo diferentes dos esperados. Na maioria dos cenários, o HTTP/2 apresentou desempenho pior do que o HTTP/1.1, e além disso técnicas que se esperava que não fossem alterar o comportamento das páginas \textit{web} no novo protocolo, acabaram apresentando mudanças no tempo de carregamento.

Muitos podem ser os motivos para os resultados encontrados. Dentre eles, Tsujikawa, o desenvolvedor do \textit{nghttp2}, acredita que os navegadores ainda não estão preparados para utilizar todo o potencial do novo protocolo. Além disso, como o HTTP/2 aumenta o paralelismo dos \textit{downloads} o Nginx pode não estar conseguindo servir todas as requisições ao mesmo tempo ou até mesmo o processador do servidor de teste pode não suportar tanto paralelismo. Outra causa possível para esse resultado é que o paralelismo dos \textit{downloads} é limitado pela conexão à Internet, velocidade ou largura de banda.

Contudo o HTTP/2 apresentou maior estabilidade no tempo de carregamento das páginas, o que pode ser percebido pela análise do desvio padrão, o que já pode ser considerado um avanço, considerando que isso pode garantir que os desenvolvedores possam prever melhor o comportamento de seus usuários.

Vale ressaltar que o HTTP/2 foi aprovado em Março de 2015 e que muito ainda tem a evoluir. Muitas de suas funcionalidades ainda não foram implementadas e os servidores ainda não foram otimizados para trabalhar com essa nova versão do protocolo. Espera-se que a medida que o procolo vá avançando e seja adotado pelos gigantes da Internet, novos avanços sejam feitos.

\section{Principais contribuições}
\label{principaiscontribuicoes}

Ainda que os resultados esperados não tenham sido obtidos, este trabalho é uma importante ferramenta de comparação entre as diferentes versões do protocolo HTTP e de explicação do que se esperar no novo protocolo HTTP/2. Não obstante, ele demonstra maneiras errôneas de se configurar o módulo \textit{mod\_h2}, que será o módulo oficial para o HTTP/2 no servidor Apache, que podem evitar que novas tentativas de configuração falhem.

Mas além das falhas obtidas, este trabalho apresenta uma maneira detalhada de como se configurar o \textit{nghttp2} e de como utilizar o HTTP/2 no Nginx, com a instalação e configuração da versão 1.9.5. Com isso mais pessoas podem analisar o desempenho do novo protocolo para ajudar a responder as perguntas em aberto e entender melhor como todo o potencial do HTTP/2 pode ser utilizado.


\section{Trabalhos futuros}
\label{trabalhosfuturos}

Tendo em vista os esforços feitos para se definir uma nova versão para o protocolo HTTP, acredita-se que existirá um esforço da comunidade para a implantação e expansão do mesmo. Muitas perguntas ficam em aberto com relação ao comportamento do HTTP/2 e o que tem de ser feito para se obter os resultados desejados com esse novo protocolo. Como trabalho futuro fica a proposta de se fazer uma análise de qual componente está limitando que o maior paralelismos gere ganhos no tempo de carregamento das páginas \textit{web}. Além disso, técnicas que não deveriam gerar mudanças no desempenho das páginas no novo protocolo, acabaram tendo o mesmo efeito que o HTTP/1.1. Não se sabe o motivo disso, mas a medida que novas atualizações forem surgindo para o HTTP/2 pode-se esperar que o comportamento real se aproxime mais da proposta teórica, então será válido fazer uma nova análise das técnicas propostas por Steve Souders.