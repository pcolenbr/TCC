%
% Documento: Descrição do problemas
%

\chapter{Descrição do problema}

Com a evolução da Internet os \textit{websites} passaram a ser cada vez mais robustos e volumosos. Foram inventadas novas maneiras de inserir informações, criar interações e estilizar as páginas da web. Com o surgimento de linguagens como o CSS e o JavaScript, os \textit{sites} ficaram mais atraentes e interessantes e, além disso, eles deixaram de ser apenas páginas estáticas para o compartilhamento de conteúdo e passaram a ser aplicações complexas com várias funcionalidades. Com esses novos sites iterativos e atraentes surgiu também a necessidade de técnicas para torná-los mais rápidos.

Ao longo dos anos técnicas para melhorar o desempenho dos \textit{websites} foram desenvolvidas e aplicadas em muitas páginas da Web. Mas o protocolo de transferência de hipermídia, o HTTP, está sofrendo mudanças e já foi confirmada o lançamento de uma nova versão, o HTTP2. Com esse novo protocolo deverão ocorrer mudanças na maneira como são feitas as otimizações de desempenho dos \textit{websites}.

Este trabalho tem a proposta de avaliar a necessidade e a eficácia das técnicas existentes com a chegada do HTTP2 e ainda propor novas técnicas adequadas ao novo portocolo caso isso seja necessário.
