\begin{quadro}[!htb]
	\centering
	\caption{Códigos de estado HTTP.\label{qua:estadoshttp}}
	\begin{tabularx}{\textwidth}{| X | X | X |}
		\hline
		\textbf{Código} & \textbf{Significado} & \textbf{Exemplo}                            \\
		\hline
		1xx             & Informação           & 100 = servidor concorda em lidar com requisição do cliente       \\
		\hline
		2xx             & Sucesso              & 200 = sucesso na requisição; 204 = nenhum conteúdo presente      \\
		\hline
		3xx             & Redirecionamento     & 301 = página foi movida; 304 = \textit{cache} ainda é válida       \\
		\hline
		4xx             & Erro do cliente      & 403 = página proibida; 404 = página não encontrada               \\
		\hline
		5xx             & Erro do servidor     & 500 = erro interno de servidor; 503 = tente novamente mais tarde \\
		\hline
	\end{tabularx}
	\fonte{Adaptado de \citeonline{Tanenbaum}}
\end{quadro}