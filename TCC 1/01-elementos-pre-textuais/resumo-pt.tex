%
% Documento: Resumo (Português)
%

\begin{resumo}

Desde a década de 1990, a \textit{Web} não para de se expandir e adquirir novas funcionalidades. Nos dias atuais, a parte mais popular da Internet já não é mais composta de páginas estáticas que serviam apenas para leitura. Juntamente com a evolução da \textit{Web}, a quantidade de dados que trafega entre servidores e usuários finais aumentou consideravelmente, isso graças à complexidade das páginas \textit{web}. Um aumento do tempo de carregamento das páginas resulta em usuários insatisfeitos e prejuízos para empresas. Para diminuir esse tempo de carregamento, Steve Souders dedicou sua carreira em encontrar técnicas para otimização de desempenho de \textit{front-end} de \textit{websites}. Suas técnicas se provaram eficientes para o HTTP/1.1, que foi a versão vigente do protocolo HTTP por muitos anos. Com a iminência da liberação do HTTP/2, não se sabe se as técnicas de Souders continuarão funcionando para a nova versão do protocolo. Este trabalho analisou as técnicas propostas por Steve Souders, encontrou quais poderiam ser afetadas pelas mudanças do protocolo HTTP e as testou na nova versão. O resultado obtido a partir deste trabalho é uma lista de técnicas de otimização de desempenho para \textit{front-end} de \textit{websites} que funcionam no protocolo HTTP/2.

\end{resumo}
