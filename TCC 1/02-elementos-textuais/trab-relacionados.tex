%
% Documento: Trabalhos Relacionados
%

\chapter{Trabalhos Relacionados}
Desenvolvedores estão sempre a procura de maneiras para tornar suas aplicações mais rápidas aos olhos dos usuários finais e isso não é diferente com os \textit{websites}. A área de otimização de desempenho para páginas \textit{web} é tão antiga quanto a \textit{World Wide Web}, mas por muito tempo os esforços foram dedicados à otimização de \textit{back-end}, o que \cite{HighPerformance} provou não ser o melhor caminho para se atingir o objetivo desejado. De acordo com \citeonline[p.~5]{HighPerformance}: "Apenas 10-20\% do tempo de resposta do usuário final são gastos baixando o documento HTML. Os outros 80-90\% são gastos baixando todos os componentes da página.". Então, os desenvolvedores podem obter melhores resultados de desempenho caso foquem seus esforços na otimização do \textit{front-end} de seus \textit{websites} ao invés do \textit{back-end}.

Neste capítulo serão apresentadas as ideias principais dos dois livros  de Steve Souders. Esses livros tornaram o conhecimento sobre a otimização de desempenho de \textit{front-end} de \textit{websites} acessível a todos os desenvolvedores e ajudaram a melhorar a \textit{web} nos últimos anos.

\section{\textit{High Performance Web Sites}}
\label{sec:highperformancewebsites}

No livro \textit{High Performance Web Sites} \cite{HighPerformance}, Steve Souders descreve 14 regras para a otimização de desempenho de \textit{front-end} de \textit{websites}. De acordo com \citeonline[p.~5]{HighPerformance}: "Se você seguir todas as regras aplicáveis a seu \textit{website}, você vai fazer sua página 25-50\% rápida e melhorar a experiencia do usuário.". As regras são muito variadas e contemplam de configurações básicas de servidores, a redução do tamanho de arquivos e melhores práticas de programação.

A seguir, serão explicadas as 14 regras encontradas no livro, ordenadas, de acordo com Steve Souders, das que causam maior impacto no desempenho às que causam menor impacto.

\subsection{Regra 1: Faça menos requisições HTTP}
\label{subsec:highperformance_regra1}
Cada componente de uma página \textit{web} gera pelo menos uma requisição e uma resposta HTTP para chegar do servidor ao cliente. Essas chamadas HTTP costumam ser um grande gargalo de desempenho para \textit{websites} e aplicações \textit{web}, pois como explicou Souders, 80-90\% do tempo de montagem dessas páginas é gasto baixando outros componentes além do HTML. Assim uma solução para diminuir o tempo de resposta para o usuário final é diminuir o número de componentes que precisam ser baixados. Com esse objetivo, as seguintes técnicas devem ser utilizadas:
	\begin{itemize}
		\item CSS \textit{Sprites} \footnote{http://www.w3schools.com/css/css\_image\_sprites.asp}
		\item \textit{Inline images}
		\item Combinar \textit{scripts} e folhas de estilo \footnote{A técnica não sugere a união de \textit{scripts} com folhas de estilo, e sim \textit{scripts} com \textit{scripts} e folhas de estilo com folhas de estilo, diminuindo o número de arquivos existentes.}
	\end{itemize}

\subsection{Regra 2: Use Redes de Entrega de Conteúdo (CDN)}
\label{subsec:highperformance_regra2}
No HTTP/1.1 o número de conexões TCP que os navegadores \textit{web} abrem em paralelo no mesmo domínio é limitado, com o objetivo de garantir a qualidade de cada conexão. Apesar desse número ser maior que no HTTP/1.0, o limite continua existindo. Outra característica dessas conexões é que elas são afetadas pela distância física entre clientes e servidores, logo, um usuário que esteja no Brasil vai demorar mais para acessar componentes que estejam hospedados na China do que componentes que estejam hospedados nos Estados Unidos. Uma maneira de resolver esses dois problemas simultaneamente são as CDNs \footnote{http://en.wikipedia.org/wiki/Content\_delivery\_network}. Ao invés de hospedar todos os componentes do \textit{website} no mesmo servidor, os desenvolvedores deveriam utilizar das CDNs para garantir maior paralelismo e diminuir as distâncias geográficas entre seus sistemas e seus usuários.

\subsection{Regra 3: Adicione cabeçalhos de expiração}
\label{subsec:highperformance_regra3}
O HTTP foi desenvolvido com suporte nativo para gerenciamento de \textit{cache}. Mas para essa \textit{cache} funcionar de maneira eficiente é necessário que os desenvolvedores configurem seus servidores para informar aos navegadores \textit{web} que certos arquivos devem ficar salvos localmente. A maneira de garantir que os arquivos serão salvos em \textit{cache}, é o envio da etiqueta \textit{Expires} no cabeçalho HTTP da resposta que possui o arquivo. Essa configuração não é definida por padrão nos servidores, então os desenvolvedores devem analisar quais arquivos devem utilizar desse recurso e por quanto tempo eles serão válidos. A \textit{cache} é uma ferramente muito importante na otimização de desempenho, e deixar de usa-la causa grandes danos no tempo médio de resposta de \textit{websites} e aplicações \textit{web}.

\subsection{Regra 4: Utilize \textit{gzip} no componentes}
\label{subsec:highperformance_regra4}
De acordo com \citeonline[p.~29]{HighPerformance} essa é a técnica mais simples e que causa maior efeito no tamanho em \textit{bytes} das páginas \textit{web}. O HTTP e os navegadores \textit{web} possuem suporte para compressão de todos os tipos de arquivos, os desenvolvedores só precisam garantir que os clientes e os servidores informem uns aos outros que a esse mecanismo está habilitado. Para fazer isso, a etiqueta de cabeçalho \textit{Accept-Encoding} deve ser definida com os formatos de compressão suportados. O \textit{gzip} é um formato de compressão criado especialmente para a \textit{web} e é o mais eficiente para ser utilizado no HTTP.

\subsection{Regra 5: Coloque folhas de estilo no topo da página}
\label{subsec:highperformance_regra5}
Como disse \citeonline[p.~38]{HighPerformance}, "essa regra tem menos a ver com melhorar o tempo de carregamento da página e mais a ver com como o navegador \textit{web} reage à ordem dos componentes.". Por definição, os navegadores não mostram os elementos da tela até que todas as folhas de estilo tenham sido carregadas, para evitar erros de desenho. Sendo assim, os desenvolvedores devem colocar as folhas de estilo no topo da página (de preferência, dentro do elemento HTML \textit{<head></head>}\footnote{http://www.w3schools.com/tags/tag\_head.asp}), pois elas devem ser os primeiros componentes a serem baixados.

\subsection{Regra 6: Coloque \textit{scripts} no fim da página}
\label{subsec:highperformance_regra6}
Ao contrário das folhas de estilo, os arquivos de \textit{script} devem ser colocados no final do arquivo HTML. Componentes localizados abaixo de arquivos de \textit{script} são bloqueados de serem baixados e desenhados até que o \textit{script} tenha terminado de ser carregado. Isso porque o navegador quer garantir que o \textit{script} esteja pronto caso ele vá executar alguma alteração no restante da página.

\subsection{Regra 7: Evite expressões CSS}
\label{subsec:highperformance_regra7}
Apesar de já não serem suportadas na maioria dos navegadores \textit{web}, os desenvolvedores não devem utilizar expressões CSS em nenhum caso, mesmo nos casos em que o navegador dá essa alternativa. Este tipo de expressão é calculada toda vez que há uma movimentação na página (desenhos de novos componentes, uso da barra de rolagem, dentre outros) e isso pode afetar, em muito, a performance de um \textit{website} ou aplicação \textit{web}.

\subsection{Regra 8: Faça arquivos \textit{JavaScripts} e folhas de estilo externos}
\label{subsec:highperformance_regra8}
Existem duas maneiras de incluir estilos e \textit{scripts} em um \textit{website} ou em uma aplicação \textit{web}. A primeira delas é inserir blocos de código em linha, ou seja, colocar os código diretamente no arquivo HTML, de preferência seguindo as Regras 5 e 6. Apesar de funcionar, essa maneira impede um bom uso da \textit{cache}, porque arquivos HTML geralmente não são salvos em \textit{cache}, logo, os estilos e \textit{scripts} têm de ser baixados todas as vezes que a página é carregada. Levando em consideração que a Regra 3 está sendo usada, uma maneira mais eficiente se inserir estilos e \textit{scripts} em uma página é colocando-os em arquivos externos e adicionando os arquivos no HTML. Dessa forma os arquivos externos podem ser salvos em \textit{cache} e não precisam ser baixados todas as vezes que o HTML é requisitado.

\subsection{Regra 9: Reduza o número de pesquisas de DNS}
\label{subsec:highperformance_regra9}
Usar diferentes domínios para hospedar componentes é uma maneira de possibilitar que esses componentes sejam baixados em paralelo. Mas a pesquisa por domínios é cara e pode acabar influenciando no tempo de carregamento de uma página. Por isso os navegadores salvam os domínios que já sabem em \textit{cache}. Levando em consideração esse custo de pesquisa, existe um limite para até quando hospedar componentes em domínios diferentes é benéfico. Os desenvolvedores têm que analisar se a quantidade de pesquisas de DNS para domínios únicos (que não estão em \textit{cache}) estão sendo feitas no carregamento da página e decidir se ainda vale a pena aumentar o paralelismo. Como regra geral, os desenvolvedores deveriam distribuir seus componentes em pelo menos dois e no máximo quatro domínios.

\subsection{Regra 10: Minimizar arquivo \textit{JavaScript}}
\label{subsec:highperformance_regra10}
Minimizar é a técnica de reduzir caracteres desnecessários de códigos. Essa técnica pode ser utilizada para qualquer tipo de arquivo CSS, JS, HTML, dentre outros. A minimização reduz o tempo de carregamento de páginas \textit{web} porque reduz o tamanho do arquivo diminuindo o tempo gasto para baixa-lo.

\subsection{Regra 11: Evite redirecionamentos}
\label{subsec:highperformance_regra11}
Redirecionamentos são situações onde o usuário tenta acessar uma página \textit{web} e essa página o informa que ele deve ser encaminhado para outra página. Os redirecionamentos são feitos através dos códigos da família 3xx do protocolo HTTP, sendo que os diferentes códigos dessa família informam a razão para o redirecionamento.

Redirecionamentos são extremamente danosos para a experiência de uso de \textit{websites} e aplicações \textit{web}, pois nada é mostrado para o usuário até que o redirecionamento seja concluído e o conteúdo HTML termine de ser carregado. Logo, desenvolvedores não devem utilizar redirecionamento, a não ser em casos em que são indispensáveis.

\subsection{Regra 12: Remova \textit{scripts} duplicados}
\label{subsec:highperformance_regra12}
Em projetos grandes, é comum que vários desenvolvedores mexam no código e isso pode fazer com que o mesmo código seja inserido mais de uma vez. \textit{Scripts} duplicados aumentam o número de chamadas HTTP e isso é muito prejudicial ao desempenho de \textit{websites} e aplicações \textit{web}. Sendo assim os desenvolvedores devem ser muito cuidadosos quanto aos \textit{scripts} que estão sendo carregados em suas páginas e ter certeza de que todos são únicos e necessários.

\subsection{Regra 13: Configure \textit{ETags}}
\label{subsec:highperformance_regra13}
Muitas vezes os desenvolvedores não configuram como funcionará o mecanismo de \textit{ETags} de seus servidores e isso pode fazer com que o mecanismo de \textit{cache} seja prejudicado. Como o valor do \textit{ETag} de um componente é único para cada servidor, se o cliente requisitar um componente no servidor X e mais tarde requisitar o mesmo componente no servidor Y, mesmo que este tenha sido salvo na \textit{cache} do navegador na primeira vez, ele será baixado novamente. Na maioria das vezes é melhor desabilitar a opção de \textit{ETags} do que correr o risco de influenciar negativamente no sistema de \textit{cache}.

\subsection{Regra 14: Habilite \textit{cache} para AJAX}
\label{subsec:highperformance_regra14}
As chamadas AJAX melhoram muito a experiência do usuário, pois tornam as páginas \textit{web} mais dinâmicas e responsivas. O problema é que muitas vezes elas não são salvas em \textit{cache} e isso pode afetar o desempenho dos \textit{websites} e aplicações \textit{web}. Para garantir que as chamadas AJAX serão salvas em \textit{cache} os desenvolvedores devem seguir seguir as regras 1 à 13 também para esse tipo de componente, tomando cuidado especial com a Regra 3.

\section{\textit{Even Faster Web Sites}}
\label{sec:evenfasterwebsites}

Com o objetivo de difundir ainda mais as técnicas de otimização para \textit{front-end} de \textit{websites}, Steve Souders lançou, em 2009, seu segundo livro, intitulado \textit{Even Faster Web Sites}. Nesta obra, Souders teve a contribuição de outros 8 profissionais da área de otimização que escreveram 6 dos 14 capítulos do livro.

As técnicas descritas em \textit{Even Faster Web Sites} são mais avançadas, e, consequentemente, mais difíceis de serem aplicadas do que as da obra anterior. Além disso, elas não vêm em formato de regras a serem seguidas, o que facilitava a aplicação das regras descritas em \textit{High Performance Web Site}. Alguns capítulos do livro explicam melhores técnicas de programação, descrevem o funcionamento de mecanismos da \textit{web} e outros ainda reforçam ideias anteriores.

Nesta seção serão resumidas as ideias principais de cada capítulo.

\subsection{Entendendo performance em AJAX}
\label{subsec:evenfaster_cap1}
Para saber se vale a pena otimizar o desempenho de um \textit{website} ou de uma aplicação \textit{web}, os desenvolvedores devem levar em conta o tempo e o esforço que deverão ser empregados no processo de otimização, e conhecer quanto essa otimização pode melhorar a experiência do usuário. Desta forma, pode-se decidir se compensa fazer um investimento nesse processo. O desenvolvedor deve se lembrar que otimizar componentes que não têm contribuição expressiva no tempo de carregamento de uma página \textit{web} não é muito relevante para a experiencia do usuário.

Os componentes AJAX representam parte importante das aplicações na \textit{Web} 2.0. Esses componentes são carregados enquanto o usuário está navegando na página, então deve-se evitar qualquer cenário em que a aplicação congele à espera da resposta de uma requisição AJAX. Os desenvolvedores devem identificar quais componentes compensam ser otimizados utilizando os "Eixos do Erro", descrito por \citeonline[p.~2]{EvenFaster} e focar seus esforços nesse componentes.

Durante o processo de otimização de componentes AJAX, os desenvolvedores devem se lembrar:
\begin{itemize}
	\item A maior parte do tempo do navegador é gasta na construção do DOM e não no \textit{JavaScript}.
	\item Códigos organizados são mais fáceis de serem otimizados.
	\item Iterações afetam muito o desempenho de uma aplicação \textit{web}.
	\item Truques (\textit{tricks}) não compensam, a não ser que eles se provem realmente eficientes.
\end{itemize}

\subsection{Criando aplicações \textit{web} responsivas}
\label{subsec:evenfaster_cap2}
Aplicações responsivas são aquelas que respondem ao usuário de maneira rápida e eficiente, fazendo com que ele não sinta que está esperando. Atrasos maiores do que 0,2 segundos causam no usuário a sensação de que o navegador está tendo problemas para conseguir sua resposta, e isso afeta sua experiência de uso. O mecanismo responsável pela responsividade das aplicações é o AJAX.

Em aplicações grandes, muitas chamadas AJAX pode ser executadas ao mesmo tempo e isso é um problema para o \textit{JavaScript}. De acordo com Brendan Eich, criador do \textit{JavaScript}, sua linguagem não possui e nunca vai possuir \textit{threads}. Assim fica por responsabilidade dos desenvolvedores encontrar técnicas para melhorar o paralelismo de suas aplicações.

\subsection{Dividindo carga inicial}
\label{subsec:evenfaster_cap3}
O tempo inicial de carregamento de uma página é um fator muito importante na otimização de desempenho. Usuários não gostam de esperar para poder interagir com uma aplicação ou \textit{website} e a grande maioria desses usuários desiste de esperar se o tempo for muito longo. Sendo assim, os desenvolvedores devem definir quais métodos de seus \textit{scripts} são necessários no evento evento \textit{onload} (evento executado assim que a página é carregada) e separa-los em arquivos diferentes dos outros métodos. Os métodos que devem ser executados logo que a página for carregada devem ser declarados no início do documento HTML. Os outros podem ser declarados ao final da página ou até mesmo de maneira assíncrona.

\subsection{Carregando \textit{scripts} sem bloqueios}
\label{subsec:evenfaster_cap4}
\textit{Scripts} possuem um efeito muito negativo no carregamento de páginas. Enquanto estão sendo baixados e executados, eles bloqueiam o carregamento de componentes localizados abaixo deles. Então, é muito importante definir quais \textit{scripts} podem ser baixados independentes do resto da página e encontrar maneiras de requisitar esses \textit{scripts} sem bloquear o resto da página. No Capítulo 4 deste livro, \citeonline{EvenFaster} descreve seis técnicas de como carregar arquivos de \textit{script} externos de forma assíncrona:

\begin{itemize}
	\item \textit{XHR Eval}, \cite[p.~29]{EvenFaster}
	\item \textit{XHR Injection}, \cite[p.~31]{EvenFaster}
	\item \textit{Script in Iframe}, \cite[p.~31]{EvenFaster}
	\item \textit{Script in DOM Element}, \cite[p.~32]{EvenFaster}
	\item \textit{Script Defer}, \cite[p.~32]{EvenFaster}
	\item \textit{document.write Script Tag}, \cite[p.~33]{EvenFaster}
\end{itemize}

Não existe uma solução única para todos os \textit{websites} e aplicações \textit{web}, então os desenvolvedores devem analisar qual é a melhor para sua situação específica. E caso seja definido que a página terá de ser congelada por algum tempo, os desenvolvedores devem se certificar de que algum indicador de navegador ocupado está sendo mostrado ao usuário\footnote{http://www.stevesouders.com/blog/2013/06/16/browser-busy-indicators/}.

\subsection{Lidando com \textit{scripts} assíncronos}
\label{subsec:evenfaster_cap5}
Quando as técnicas descritas no Capítulo 4 \cite[p.~27]{EvenFaster}, são utilizadas para carregar \textit{scripts} sem bloqueios, cria-se um novo problema. Componentes que são carregados de maneira assíncrona estão sujeitos a condições de corrida\footnote{http://en.wikipedia.org/wiki/Race\_condition}. Essas condições tornam imprevisível a ordem na qual os \textit{scripts} serão carregados, o que pode fazer com que ocorram falhas nas dependências de funções. Para evitar essas falhar, Steve Souders descreve 5 técnicas de como garantir que os \textit{scripts} serão carregados em uma ordem na qual serão executados sem erros de dependências:

\begin{itemize}
	\item \textit{Hardcoded Callback}, \cite[p.~46]{EvenFaster}
	\item \textit{Windows Onload}, \cite[p.~47]{EvenFaster}
	\item \textit{Timer}, \cite[p.~48]{EvenFaster}
	\item \textit{Script Onload}, \cite[p.~49]{EvenFaster}
	\item \textit{Degrading Script Tags}, \cite[p.~50]{EvenFaster}
\end{itemize}

Ao final do capítulo, Souders ainda descreve uma técnica que chamou de "Solução Geral" \cite[p.~59]{EvenFaster}.

\subsection{Posicionando blocos de \textit{scripts} em linha}
\label{subsec:evenfaster_cap6}
Apesar de não gerarem novas requisições HTTP, blocos de \textit{scripts} em linha, ou seja, aqueles posicionados dentro do documento HTML, bloqueiam outros componentes de serem carregados enquanto eles estão sendo executados e isso impede o desenho progressivo da página. Para evitar esse efeito, blocos de \textit{scripts} em linha devem ser movidos para o final da página e, se possível, serem executados de maneira assíncrona. Outro grave problema com esse tipo de \textit{script} é que, quando eles são antecedidos por folhas de estilo, eles não começam a executar enquanto os estilos não são carregados. Dessa forma, é uma boa técnica não colocar blocos de \textit{scripts} em linha logo após chamadas de folhas de estilo.

\subsection{Escrevendo \textit{JavaScripts} eficientes}
\label{subsec:evenfaster_cap7}
Neste capítulo, Nicholas Zakas descreve boas técnicas de programação para \textit{JavaScripts} que visam otimizar o desempenho do código. As técnicas descritas são:

\begin{itemize}
	\item Melhor gerenciamento de escopos, \cite[p.79]{EvenFaster}
	\item Use variáveis locais, \cite[p.~81]{EvenFaster}
	\item Otimize acesso à dados frequentes, \cite[p.~85]{EvenFaster}
	\item Use condicionais mais eficientes, \cite[p.~89]{EvenFaster}
	\item Use iterações mais eficientes, \cite[p.~93]{EvenFaster}
	\item Otimize cadeias de caracteres, \cite[p.~99]{EvenFaster}
	\item Evite função com tempo de execução longo, \cite[p.~102]{EvenFaster}
	\item Habilite contadores de tempo, \cite[p.~103]{EvenFaster}
\end{itemize}

\subsection{Escalando usando \textit{Comet}}
\label{subsec:evenfaster_cap8}
Quando muitos dados precisam ser entregues de maneira assíncrona, o mecanismo de AJAX pode acabar causando um problema conhecido como \textit{long-polling} de requisições. \textit{Long-polling} ocorre quando muitas chamadas ao servidor são feitas em um curto intervalo de tempo e o servidor não consegue responder todas no tempo desejado, então algumas dessas chamadas falham, ou ficam muito tempo paradas. \textit{Comet} é um termo que se refere a uma coleção de técnicas, protocolos e implementações que tem como objetivo melhorar o tráfego de dados em baixa latência \footnote{http://ajaxexperience.techtarget.com/images/Presentations/Carter\_Michael\_ScalableComet.pdf}.

\subsection{Indo além do \textit{gzip}}
\label{subsec:evenfaster_cap9}
Mesmo com o suporte à compressão de dados com \textit{gzip} habilitado nos \textit{websites} e aplicações, uma média de 15\% dos usuários continua recebendo dados sem compressão alguma. A maior causa disso são os \textit{proxies} e os programas de anti-vírus. Estes sistemas modificam os cabeçalhos HTTP para poderem ter acesso aos dados que estão sendo enviados a fim de realizarem suas tarefas de inspeção do conteúdo trafegado.

Se essas entregas sem compressão fossem dissolvidas entre todos os usuários da página \textit{web}, o problema seria menos relevante. Mas o que ocorre é que os mesmos 15\% dos usuário estão sempre acessando os \textit{websites} e aplicações de forma mais lenta, o que acaba com a experiência desses usuários e os desencoraja a voltar a acessar a página.

Considerando este fato, Tony Gentilcore, reforça as ideias expostas por Steve Souders no livro \textit{High Performance Web Sites}, \cite{HighPerformance}, de que os desenvolvedores devem fazer seus arquivos os menores possíveis e reduzir o número de chamadas HTTP.

\subsection{Otimizando imagens}
\label{subsec:evenfaster_cap10}
Imagens são consideradas ótimos componentes para se conseguir melhora de desempenho sem ter de abrir mão de funcionalidades. Existem vários padrões de formato para imagens\footnote{http://en.wikipedia.org/wiki/Image\_file\_formats}, cada um deles é determinado pelo tipo de compressão que usa e cada um deles possui seus prós e contras. Sendo assim, os desenvolvedores devem decidir que tipo de qualidade que eles desejam em suas imagens, então selecionar o formato que se adeque às suas necessidades e otimizar essas imagens o máximo possível.

Como regra geral, os desenvolvedores devem optar por utilizar o formato PNG sempre que possível.

\subsection{Quebrando domínios dominantes}
\label{subsec:evenfaster_cap11}
Mesmo que isso aumente a quantidade de pesquisas de DNS, muitas vezes aumentar o número de domínios nos quais componentes estão hospedados melhora o desempenho de \textit{websites} e aplicações \textit{web}. O motivo disso é que mais componentes podem ser baixados em paralelo, o que diminui o tempo total de carregamento da página. O desafio é encontrar o número de domínios que resulta no menor tempo de carregamento da página. Como regra geral, \citeonline[p.~168]{EvenFaster} determina que dividir componentes entre dois a quatro domínios gera resultados satisfatórios.

\subsection{Entregando o documento cedo}
\label{subsec:evenfaster_cap12}
Por padrão, uma página \textit{web} só começa a baixar os seus componentes depois que o documento HTML é totalmente carregado. Sendo assim, durante algum tempo, a única atividade que está sendo realizada pelo navegador é a espera do carregamento do documento HTML. Contudo, existem técnicas para diminuir esse tempo de espera e, consequentemente, o tempo de carregamento de uma página.

Uma boa prática é entregar o documento HTML em partes para o navegador, desde que essas partes possam acelerar o processo de carregamento. Para isso, Steve Souders sugere que o cabeçalho HTML, onde normalmente são declaradas as folhas de estilo, seja entregue assim que estiver pronto. Dessa forma, o navegador pode começar a baixar esses componentes antes mesmo que o carregamento do HTML esteja concluído. Essa técnica é possível graças à etiqueta \textit{Chuncked Encoding}, adicionada no HTTP/1.1, e às funções de descarregamento de linguagens de \textit{back-end} como PHP e NodeJS.

\subsection{Usando \textit{Iframes} com moderação}
\label{subsec:evenfaster_cap13}
Os \textit{Iframes} são componentes HTML que permitem incorporar documentos dentro de outros documentos. Apesar de terem sido muito utilizados no passado, principalmente para adicionar publicidades a \textit{websites}, os \textit{Iframes} são pouco usados nos dias de hoje. A razão disso é que são componentes muito pesados e que bloqueiam o desenho das páginas \textit{web}.

Como regra geral, Steve Souders aconselha não utilizar estes componentes.

\subsection{Simplificando seletor CSS}
\label{subsec:evenfaster_cap14}
Na \textit{Web} 2.0, folhas de estilo são tão populares quanto \textit{scripts}. \textit{Websites} possuem várias folhas de estilo, mas pouco esforço é empregado em otimizar esses arquivos.

Neste capítulo, Souders explica que os navegadores fazem buscas por componentes declarados em folhas de estilo da direita para esquerda, logo os identificadores mais à direita devem ser os mais específicos, para diminuir o tempo de busca. Além disso ele alerta para o mal de códigos duplicados e cita boas práticas de programação para CSS, \cite[p.~195]{EvenFaster}.
